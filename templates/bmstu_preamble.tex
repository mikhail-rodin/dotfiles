\documentclass[12 pt]{article}
\usepackage{amsmath,amsthm,amssymb}
\usepackage{mathtext}
\usepackage[T1,T2A]{fontenc}
\usepackage[utf8]{inputenc}
\usepackage[english,russian]{babel}
\usepackage{indentfirst}
\usepackage{csquotes}
\usepackage{siunitx}
\usepackage{textcomp}

\usepackage{xfrac}

\usepackage[parentracker=true,
backend=biber,
hyperref=false,
bibencoding=utf8,
style=numeric-comp,
language=auto,
autolang=other,
citestyle=gost-numeric,
defernumbers=true,
bibstyle=gost-numeric,
sorting=ntvy,
]{biblatex}

\usepackage{enumitem}
\setlist[itemize]{noitemsep, nolistsep}

\usepackage{graphicx}
\usepackage[export]{adjustbox}

\usepackage{accents}
\usepackage{mathtools}
\usepackage{bm}
\usepackage{esvect}
\usepackage{gnuplottex}
\usepackage[activate={true,nocompatibility},final,tracking=true,kerning=true,spacing=true,factor=1100,stretch=10,shrink=10]{microtype}

\usepackage[makeroom]{cancel}
%\cancelto{\infty}{2 y} - diagonal strikeout

%flexible strikeout:
%\usepackage{xparse}
%\DeclareDocumentCommand{\hcancel}{mO{0pt}O{0pt}O{0pt}O{0pt}}{%
%    \tikz[baseline=(tocancel.base)]{
%        \node[inner sep=0pt,outer sep=0pt] (tocancel) {#1};
%        \draw[red] ($(tocancel.south west)+(#2,#3)$) -- ($(tocancel.north east)+(#4,#5)$);
%    }%
%}%
%syntax \hcancel{<text>}{<start. point horiz. shifting>}{<start. point vertical shifting>}{<end. point horiz. shifting>}{<end. point vertical shifting>}

\usepackage{relsize}
% \text{\Large{$G$}} - font size in math mode

%Figure placing:
\usepackage{float}
\usepackage{wrapfig}
%h	Place the float here, i.e., approximately at the same point it occurs in the source text (however, not exactly at the spot)
%t	Position at the top of the page.
%b	Position at the bottom of the page.
%p	Put on a special page for floats only.
%!	Override internal parameters LaTeX uses for determining "good" float positions.
%H	Places the float at precisely the location in the LATEX code. Requires the float package, though may cause problems occasionally. This is somewhat equivalent to h!.
%команда eqref доступна для ссылки на формулы

%\begin{wrapfigure}{r}{0.25\textwidth} %this figure will be at the right

%\setlength\intextsep{0pt} - allow no unused vertical space 
%\intextsep The vertical space placed above and below a float that is put in the middle of the text with the h location option. It is a rubber length.

\usepackage{tikz}
%\begin{tikzpicture}<br>
%    \draw[help lines] (0,0) grid (3,2);<br>
%    \draw (0,0) coordinate (A) -- (3,2) coordinate (B)<br>
%          (1,2) -- (3,0);<br>
%    \fill[red] (intersection of A--B and 1,2--3,0) circle (2pt);<br>
%  \end{tikzpicture}
\usetikzlibrary{3d, calc, math, tikzmark}

%%% Code from https://tex.stackexchange.com/a/156581/95438 %%%
%
\usepackage{keycommand}
% Patch by Joseph Wright ("bug in the definition of \ifcommandkey (2010/04/27 v3.1415)"),
% https://tex.stackexchange.com/a/35794
\begingroup
\makeatletter
\catcode`\/=8 %
\@firstofone
{
    \endgroup
    \renewcommand{\ifcommandkey}[1]{%
        \csname @\expandafter \expandafter \expandafter
        \expandafter \expandafter \expandafter \expandafter
        \kcmd@nbk \commandkey {#1}//{first}{second}//oftwo\endcsname
    }
}
%--------%
\newkeycommand{\hcancel}[hshiftstart=0pt,vshiftstart=0pt,hshiftend=0pt,vshiftend=0pt,color=red][1]{%
    \tikz[baseline=(tocancel.base)]{
        \node[inner sep=0pt,outer sep=0pt] (tocancel) {#1};
        \draw[\commandkey{color}] ($(tocancel.south west)+(\commandkey{hshiftstart},\commandkey{vshiftstart})$) --
        ($(tocancel.north east)+(\commandkey{hshiftend},\commandkey{vshiftend})$);
    }%
}%
\tikzset{cancel/.style={path picture={
\draw[#1] (path picture bounding box.south west) -- 
(path picture bounding box.north east);
}}}
\tikzset{cancelto/.style={path picture={
\draw[#1, ->] (path picture bounding box.south west) -- 
(path picture bounding box.north east);
}}}
%
%%% End of code %%%
%

\makeatletter
\DeclareRobustCommand{\cev}[1]{%
  \mathpalette\do@cev{#1}%
}
\newcommand{\do@cev}[2]{%
  \fix@cev{#1}{+}%
  \reflectbox{$\m@th#1\vec{\reflectbox{$\fix@cev{#1}{-}\m@th#1#2\fix@cev{#1}{+}$}}$}%
  \fix@cev{#1}{-}%
}
\newcommand{\fix@cev}[2]{%
  \ifx#1\displaystyle
    \mkern#23mu
  \else
    \ifx#1\textstyle
      \mkern#23mu
    \else
      \ifx#1\scriptstyle
        \mkern#22mu
      \else
        \mkern#22mu
      \fi
    \fi
  \fi
}

\newcommand{\emf}{\mathcal{E}}

\makeatother


\title{
    {\centering
    \includegraphics[width=\textwidth]{C:/Users/Rodin/Documents/TEX/bmstu.pdf}}\\
    \small{МГТУ им. Н.Э. Баумана} \\
    ПРИБОРЫ ОПТИКО-ФИЗИЧЕСКИХ ИЗМЕРЕНИЙ \\ 
    \small{ЭКЗАМЕНАЦИОННЫЕ ВОПРОСЫ} \\
    \footnotesize{ВОПРОС 10}
    } 
\author{
    \textit{6 семестр, группа РЛ2-И62Б, Родин М.И.}
    }